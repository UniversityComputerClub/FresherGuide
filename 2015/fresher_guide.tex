% HERE BE DRAGONS
% This is complicated because it is based on [SZM]'s 2012 Honours Thesis template, which in turn was based on a template written by a Physics Professor.
% ~50% of this isn't actually needed, but it's too much effort to clean it up.

\documentclass[a5paper,10pt,openany]{extreport}
\usepackage[margin=0.75in,headsep=0.25in,footskip=0.25in]{geometry}
\renewcommand{\familydefault}{\sfdefault}
%\renewcommand{\familydefault}{\ttdefault}
%\linespread{1.3}
%\usepackage{setspace}
%\onehalfspacing 
 \parskip 8pt           % sets spacing between paragraphs
 %\renewcommand{\baselinestretch}{1.5} 	% Uncomment for 1.5 spacing between lines
 %\parindent 0pt		  % sets leading space for paragraphs

\usepackage{anyfontsize}

\usepackage{mdframed}


%\usepackage{natbib}
\usepackage{makeidx}
\usepackage{graphicx}
\usepackage{caption}
%\usepackage{subfigure}
\usepackage{rotating}
%\usepackage{lscape}
\usepackage{pdflscape} % Needed for landscaping - in pdf viewer
\usepackage{verbatim}
\usepackage{amsmath, amsthm,amssymb}
\usepackage{mathrsfs}
\usepackage{hyperref}

\usepackage{epstopdf}
\usepackage{float}
\usepackage[compact]{titlesec}
\titlespacing{\chapter}{0pt}{-50pt}{0pt}
% spacing glue: how to read {12pt plus 4pt minus 2pt}
%           12pt is what we would like the spacing to be
%           plus 4pt means that TeX can stretch it by at most 4pt
%           minus 2pt means that TeX can shrink it by at most 2pt
\titlespacing\section{0pt}{0pt plus 0pt minus 14pt}{0pt plus 0pt minus 14pt}
\titleformat{\chapter}
{\normalfont\LARGE\bfseries}{\thechapter.}{1em}{}

%\usepackage[usenames,dvipsnames]{color}
%\usepackage{listings} % For code snippets
\definecolor{pink}{rgb}{1.0,0.5,0.5}
\definecolor{darkgray}{rgb}{0.1,0.1,0.1}
\definecolor{lightgray}{rgb}{0.95,0.95,0.95}
\definecolor{gray}{rgb}{0.5,0.5,0.5}
\definecolor{darkred}{rgb}{0.75,0,0}
\definecolor{darkblue}{rgb}{0,0,0.35}
\definecolor{green}{rgb}{0,0.70,0}
\definecolor{black}{rgb}{0,0,0}
%\lstset{language=Java}
%\lstset{backgroundcolor=\color{darkgray}}
%\lstset{numbers=left, numberstyle=\tiny, stepnumber=1, numbersep=5pt}
%\lstset{keywordstyle=\color{darkred}\bfseries}
%\lstset{commentstyle=\color{darkblue}}
%\lstset{stringsyle=\color{red}}
%\lstset{showstringspaces=false}
%\lstset{basicstyle=\small}

\usepackage[table]{xcolor}

\newtheorem{theorem}{Theorem}[section]
\newtheorem{lemma}[theorem]{Lemma}
\theoremstyle{definition}\newtheorem{definition}[theorem]{Definition}
\newtheorem{proposition}[theorem]{Proposition}
\newtheorem{corollary}[theorem]{Corollary}
\newtheorem{example}{Example}
\theoremstyle{remark}\newtheorem*{remark}{Remark}

\newcommand{\Phid}[0]{\dot{\Phi}}
\newcommand{\Phib}[0]{\bar{\Phi}}

\newcommand{\de}[0]{\delta}
\newcommand{\deb}[0]{\bar{\delta}}

\newcommand{\that}[0]{\hat{\theta}}

\newcommand{\vect}[1]{\boldsymbol{#1}} % Draw a vector
\newcommand{\divg}[1]{\nabla \cdot #1} % divergence
\newcommand{\curl}[1]{\nabla \times #1} % curl
\newcommand{\grad}[1]{\nabla #1} %gradient
\newcommand{\pd}[3][ ]{\frac{\partial^{#1} #2}{\partial #3^{#1}}} %partial derivative
\newcommand{\der}[3][ ]{\frac{d^{#1} #2}{d #3^{#1}}} %full derivative
\newcommand{\phasor}[1]{\tilde{#1}} % make a phasor
\newcommand{\laplacian}[1]{\nabla^2 {#1}} % The laplacian operator



% Reference things in GitHub
\newcommand{\gitref}[2]{\href{https://github.com/UniversityComputerClub/FresherGuide/master/#1/#2}{  \textcolor{black}{\emph{#2}}}}
% Refer to API commands
\newcommand{\api}[1]{  ``\textcolor{black}{\texttt{/api/#1}}''}
% Pun alert
\newcommand{\pun}[1]{#1} % Emphasising it is kind of unnecessary
% Shell command
\newcommand{\shell}[1]{\texttt{#1}}
\newcommand{\server}[1]{\texttt{#1}}
% Rants
\newcommand{\rant}[1]{\textsc{#1}}
% URL
\renewcommand{\url}[1]{<\href{#1}{\underline{\color{blue}{#1}}}>}

% Reference things.
\newcommand{\figref}[1]{(Figure \ref{#1})}
\newcommand{\seeref}[1]{(See \ref{#1})}

\newcommand{\setfont}[1]{\fontfamily{#1}\selectfont}

% To make underscores printable without escaping them
\usepackage[T1]{fontenc}
\catcode`\_=12

% Refer to code (can change each one as needed)
\newcommand{\funct}[1]{  \texttt{#1}}
\newcommand{\var}[1]{  \texttt{#1}}
\newcommand{\type}[1]{  \texttt{#1}}
\newcommand{\code}[1]{  \texttt{#1}}


%\usepackage{endfloat}
%\nomarkersintext
%\pagestyle{plain}
%\topmargin -0.6true in
%\textwidth 15true cm
%\textheight 9.5true in
%\oddsidemargin 0.25true in
%\evensidemargin 0.25true in
%\headsep 0.4true in


\usepackage{fancyheadings}
\pagestyle{fancy}
\addtolength{\headheight}{2.5pt}
\renewcommand{\chaptermark}[1]{\markboth{\thechapter~~#1}{}}
\renewcommand{\sectionmark}[1]{\markright{\thesection~~#1}{}}
\ifthenelse{\boolean{@twoside}}
{
        \rhead[\bfseries \rightmark]{\bfseries \thepage}
        \lhead[\bfseries \leftmark]{\bfseries \thepage}
        \addtolength{\headwidth}{\marginparsep}
        \addtolength{\headwidth}{\marginparwidth}
}{
        \lhead{\bfseries \leftmark}
        \rhead{\bfseries \thepage}
}
\cfoot{}

\usepackage{etoolbox}
\makeatletter
\patchcmd{\chapter}{\if@openright\cleardoublepage\else\clearpage\fi}{}{}{}
\makeatother

\newcommand{\zchapter}[1]{\chapter{#1} \setcounter{section}{-1}}

\usepackage{tocloft}
\renewcommand\cftchapfont{\small\bfseries}
\renewcommand\cftsecfont{\small}

\renewcommand\cftchappagefont{\small\bfseries}
\renewcommand\cftsecpagefont{\small}
%---------------------------------------------------------
%---------------------------------------------------------
\begin{document}

% keeps text within our margins
\sloppy

\begin{titlepage}

% Suitably pretty title page is required.

\title{UCC::FresherGuide}
\author{The University Computer Club Inc. \\ \url{http://www.ucc.asn.au} \\ <\href{mailto:ucc@ucc.asn.au}{ucc@ucc.asn.au}>}

\date{January 2014}

\maketitle
\centering

\end{titlepage}
               % This is who you are

\pagenumbering{arabic}
\pagebreak

%---------------------------------------------------------
% Do the table of Contents and lists of figures and tables
%---------------------------------------------------------
\linespread{0.3}
% Do we need these for a fresher guide?
\setcounter{tocdepth}{1}
{\small\tableofcontents}
%\listoffigures
\markboth{}{}
\linespread{1.5}

%\emph{"Interroga de Consensum Gradu Servitium nostra"}

\newpage

%\pagenumbering{arabic}
%---------------------------------------------------------
%---------------------------------------------------------
%Include the chapters!

\setcounter{chapter}{0}


\zchapter{Introduction}




% <person>'s welcome
\newenvironment{welcome}[3]
{
	\begin{mdframed}[backgroundcolor=white]
	
\color{black}{\section{#1's Welcome}}

\color{black}{Incoming message from #2  <\href{mailto:#3@ucc.asn.au}{#3@ucc.asn.au}>:}


	
}{\end{mdframed}}



\begin{welcome}{President}{Samuel Shenton}{samuel}

\%TODO: Remember to write welcome before printing

\end{welcome}

\begin{welcome}{Vice President}{Sam Moore}{matches}

Since the President hasn't written anything, it is my duty as Vice President to pick up the slack and welcome you to the 2014 Fresher Guide.

Welcome to the 2014 Fresher Guide!

Still here?

In that case, my advice to freshers is to \pun{stay fresh}.

\end{welcome}

\begin{welcome}{Fresher Representative}{Kier Campbell}{vanish}

\% TODO: Find the Fresher Representative. Is he still alive?

\end{welcome}

\begin{welcome}{Murphy}{A. C. C. Murphy}{murphy}

This year UCC turns 0x28 (40). Therefore, I have been asked to write a special welcome message.

\%TODO: Get someone who is actually funny to write Murphy's welcome.

\end{welcome}


 % Chapter 0
\zchapter{Finding the Clubroom}\label{FindClubroom}

% Directions to the clubroom, ideally with non terrible maps. Most people don't ever find it :-(


\section{Maps}


\begin{figure}[H]
	\centering
	\includegraphics[width=1\textwidth]{figures/maps.png}
	\caption{Maps} 
	\label{maps.png}
\end{figure}

\pagebreak

\section{Directions}

\begin{mdframed}

\begin{enumerate}
	\item Starting from the carpark outside The Tav, follow the arrow on the aerial picture to get to the Cameron Hall door, then go up the stairs to the second floor and follow the bottom map.
	\item If you see a coke machine and a snack machine, you are in the right place. The UCC clubroom has lots of computers and a couple of couches.
	\item If you accidentally find yourself surrounded by books and a fridge, you're in UniSFA --- try the other door in that corridor.
\end{enumerate}

You can see if the clubroom is open online at \url{http://webcams.ucc.asn.au/}

\begin{figure}[H]
	\centering
	\includegraphics[width=0.7\textwidth]{figures/webcam2.jpg}
	\caption{This is what the clubroom looks like}
	\label{webcam2.jpg}
\end{figure}


\end{mdframed}
 % Chapter 1
\zchapter{Your Account}\label{SetupAccount}

% How to actually make an account. Most Freshers never do this :-(

\begin{mdframed}

\section{Getting it}

\textsc{signing up at the O'Day stall does not give you access to all of UCC's services. You need to create an account at the UCC clubroom.}

Your UCC account is \emph{the} most important thing you can have as a member. In addition to providing a way for us to communicate with you, the account lets you log into any of our clubroom machines, as well as granting you access to our user servers, wireless network, online drink and snack machines, and more.

Once you're at the clubroom and ready to create an account, ask around for a Wheel or Committee member and have your membership sticker (we normally put it on your student card on O-Day) to set up your account. You'll need a user name and a password to memorise, but it's a pretty simple procedure once you've found the right person!


Now that you have an account, you can use it to log into any of our clubroom machines. If you want to log onto one of our servers, you'll need to use the SSH program. If you're having trouble, just ask someone in the clubroom --- we don't \pun{byte}!

Changing your UCC password can either be done by Ctrl-Alt-Del on a windows machine or using the command \shell{passwd} on a Linux/Unix machine.
Accessing the UCC WiFi network can be a bit tricky (particularly on Windows machines). Ask someone if you need help, or refer to \url{http://wiki.ucc.asn.au/Wifi}

\end{mdframed}

\pagebreak

\begin{mdframed}

\section{SSH for Great Good}

SSH is a program that lets you remotely access UCC's servers. These can be used for almost anything (legal) you can imagine; programming, website hosting, file storage, IRC chatting (see Chapter \ref{Communications}), dispensing drinks, and many more things.

The easiest way to use SSH is from a Linux clubroom desktop. Simply open a "terminal" application, type "ssh ssh" and enter your password. It should look like Figure \ref{ssh.png}.

From a windows machine, open a program called "PuTTy" (or "KiTTy" on some machines). Enter the address "ssh" and click "Open". See Figure \ref{putty.png}.

Once you are logged in, a good way to start is running "irssi -c irc" and then typing "/join \#ucc". The interface of irssi takes some getting used to, but when combined with "screen" (ask someone for help) it is a powerful weapon for procrastination.

You don't have to be in the UCC clubroom to use SSH! Simply use "username@ssh.ucc.asn.au" as the address to connect to.

\end{mdframed}

\begin{mdframed}


%\begin{figure}[H]
%	\centering
%	\includegraphics[width=1.0\textwidth]{figures/ssh.png}
%	\caption{SSH from Linux} 
%	\label{ssh.png}
%\end{figure}

\begin{figure}[H]
	\centering
	\includegraphics[width=0.5\textwidth]{figures/putty.png}
	\caption{Use PuTTY to SSH. Just press "Open"} 
	\label{putty.png}
\end{figure}

\end{mdframed}

\chapter{Dispense 101}

% How to use Dispense.

At the same time as you set up your account, you will also have your dispense account set up. Dispense is the program that allows users to store credit and purchase items from the coke/snack machines (which are almost always cheaper than what you would find elsewhere on campus).


This will involve a 5 digit User ID and a 4 digit PIN. This allows you to physically log into the snack machine and dispense both drinks and snacks.
Coke members can help you add credit to your dispense account. Call out for one in the clubroom if need be, there's always one around.


You can also use your Student Card/SmartRider as a log in device on the snack machine. To do so, log in to the Snack Machine and hold whichever card you want to use up to the card scanner (it's the thing with the blinking green light) and the card should auto- enroll. To log in using the card, simply hold the enrolled card up to the card scanner.


You can also access Dispense online via shell login and using the \shell{dispense} command. Dispense isn't installed on clubroom machines so you will have to use \shell{ssh} to access one of UCC's servers. You can get someone to show you how to do this, or alternatively, go to: \url{http://wiki.ucc.asn.au/SSH}


\chapter{UCC Events}
(dates and times subject to change)

% By [XON]

\section{UCC Fresher Welcome}
Friday, February 31st
5:00 PM
Cameron Hall Loft (above the UCC clubroom)

The Fresher welcome exists to welcome you, a new UCC member, to the club. There will be a number of current members there to talk with and get to know, and all of your questions about the club and how to use it will be answered. As a bonus, all first time members get FREE pizza.

\section{UCC Annual General Meeting}
Tuesday, March 11th
1.00 PM (common lunch hour of UWA)
Guild Council Meeting Room

The AGM is the meeting at which the new UCC committee is elected for 2014. The only way to be represented is to attend on the day. As a Fresher, you should attend to either run for or vote for the position of the Fresher Representative,who will be your liaison for the committee. If you don't know where the Guild Council Meeting Room is, arrive at the UCC clubroom a little early to join the mass exodus.

\section{Easter LAN}
Easter Weekend
3:00 PM to the morning after

UCC runs a number of LANs throughout the year, some with proper organisation, some without. The Easter LAN is the first big LAN of the academic year, taking place over the Easter weekend, the first weekend of mid-Semester one study break. We play a number of different games, and of course you can organise your own. LANs are free for all UCC members, but you can bring a friend for around \$5 (though of course you should encourage them to join). Bring your own PCs, or use one of the limited stock in the clubroom.

\section{LANs}
Throughout the Year
From Dusk til Dawn

The UCC hosts a number of more LANs throughout the year. As above, the Easter LAN is the first big one. Expect other LANs during the semester and breaks.

\section{Cameron Hall Quiz Night}
First Semester, probably in May
Evening

Bringing together the various clubs of Cameron Hall, the quiz night is the only proper time to use your smarts throughout your degree. Likely to take place in the UWA Tav, tables and tickets will be organised closer to the date.

\section{UCC Camp}
Winter Break

The UCC goes camping! Without tents. Some time during the winter break, UCC will host a camp, probably at Camp Leschenaultia. This is a chance to get your computer out of the house for a few days, tinkering and playing games with a whole bunch of other members. Don't worry, you won't be without precious internet. For first timers to the camp, some old members may offer a "scholarship" to the camp if you can't afford it. Expect more details closer to the date.

\subsection{FUCC Camp Scholarship}

A message from James Cox and Lionel Price:

For new members to UCC, Lionel and I would like to tell you about our
full-ride scholarship program. We realise that Camp is fairly
expensive, but as once-freshers made good we are financially able to
provide unto others. Previous recipients of these scholarships have
gone onto great things so we are proud to offer it once more in 2014.
I would encourage everyone who is interested to take advantage of this
offer - I had a great time my first UCC Camp which is part of the
reason why I now offer this scholarship.

Two first-time UCC Campers will have their entry fee paid for by us.
As before, you will also obtain the privilege, if you should so wish,
to add a pink F to your UCC tags, denoting sophistication, intellect,
and exclusive membership of an elite group of teamkilling imbeciles.

To be eligible for this award, you must be a UWA student, member of
UCC, and to not have attended a previous UCC Camp. Applicants will
also need to declare in writing that they will participate in at least
one game of DotA0 during UCC Camp, and that when we play ET you will
not be a noob in the back with a mortar accomplishing nothing all
match.

\section{UCC 40th Anniversary Dinner}
Saturday, September 13th
7.00 PM til late(r)
South of Perth Yacht Club

The big non-tech event of the year, the 40th Anniversary dinner is an opportunity for new members to meet the old blood of UCC, the ones that are still kicking on. Taking place in the lovely South of Perth Yacht Club, this fully catered dinner will be a good celebration of 40 years of computing. Expect entry prices to be around $60 to $80, which should cover dinner.

\section{Cameron Hall Charity Vigil}
Semester 2 - around mid-semester study break
From Dusk til Dawn

Once a year, all of the clubs in Cameron Hall get together and hold a night of fun and games to raise money for charity. While the details of the night are still to come, the UCC will probably host a LAN. There will be an entry fee for this event, but expect it to be fully worth it.

\section{UCC Tech Talks}
Throughout the Year
At Various Times

Tech talks are a chance to demonstrate your own tech-y knowledge, or learn from someone else. Taking place throughout the year, as interest demands, the talks will cover a variety of topics, previous ones including introductions to TOR, learning basics, and the magic of data compression. Early on in the semester, a number of tech talks will cover learning to use the various machines and servers of the club. 

%\zchapter{UCC Groups}

% Explanation of the groups.
\newenvironment{uccgroup}[1]
{
	\begin{mdframed}
	\section{\textsc{#1}}
	\begin{mdframed}
		Contact: <\href{mailto:#1@ucc.asn.au}{#1@ucc.asn.au}>
	\end{mdframed}
	\begin{mdframed}
		Members: \url{http://www.ucc.asn.au/infobase/groups/#1.ucc}
	\end{mdframed}

	

}{\end{mdframed}}

% Is this section a little elitist sounding, or is that just the entire group system? [SZM]
\begin{mdframed}
The UCC committee delegates specific duties and responsibilities to other people in the club. These groups, traditionally modelled after UNIX groups, are referred to often.

You can see who's in each group online (photos are sometimes included). Alternatively, if you're looking for a member of a certain group, shouting out "is there anyone here in group?" will sometimes get you an answer.


Membership of these groups entails a certain amount of trust. It is generally accepted that you will nominate yourself to the group once you feel you meet a set of requirements. This set of requirements is sometimes vague. Often it requires determined searching on the website (\url{http://www.ucc.asn.au/infobase}) to find. New group members are also often nominated by the existing group members.

% Don't really need to brand wheel like a cabal to new members. I mean, it is, but they don't need to know that for at least a few weeks.
%Members join Wheel by invite only, and will be asked to attend a Wheel Meeting, where they too will be taught the Secret Wheel Song.
%Do not despair if you're not made a Wheel member immediately. Sticking around and showing an interest through contribution is more important than just having the skills.


\end{mdframed}

\begin{uccgroup}{committee}
The Committee is appointed each year by the members (that's you) at the AGM. They handle the day-to-day running of the club, managing money, events and holding grumpy meetings each week, which members like yourself are welcome to attend.


The Fresher Rep is your voice on the Committee. Get to know them and let them know how incredibly happy you are! If you have problems, they'll always be ready to listen.

\begin{mdframed}
Fresher Rep: <\href{mailto:fresher@ucc.asn.au}{fresher@ucc.asn.au}>
\end{mdframed}
\end{uccgroup}

\begin{uccgroup}{wheel}

Wheel is in charge of maintaining the club's machines. They are the best people to see if you're having problems with the computers. Wheel maintains its own membership, but works hand in hand with Committee on issues relating to account policy. If you abuse your account, it will be locked by a Wheel member. The unlocking of accounts is at the discretion of Committee. Wheel have infrequent meetings, where they sing the secret wheel song. % THIS DOESN'T ACTUALLY HAPPEN

\end{uccgroup}

\begin{uccgroup}{door}
The Door group is responsible for the clubroom itself. Only a member of door group can unlock the clubroom and keep it open for members during the day. This means that if the only Door group member in the room has to leave, then everyone will have to leave until another Door group member arrives.
\end{uccgroup}

\begin{uccgroup}{coke}
The Coke group are the people to talk to if you want to add money to your dispense account (see the section on dispense). They can also credit your account for bad dispenses and other tasks related to dispense.
\end{uccgroup}

% Not even worth mentioning, really, it is all just wheel
\begin{uccgroup}{webmasters}

The Webmasters are charged with maintaining the UCC web presence. Becoming a Webmaster usually involves showing some interest in the UCC website. It shouldn't be too hard to get on this group.

 Please. Someone save us from the horror of doctype graham.
\end{uccgroup}

\begin{uccgroup}{winadmin}

The Winadmins group was created to give trusted members administrator access to the Windows desktop machines in the clubroom. Sprocket is the equivelant group for Linux desktop machines.
\end{uccgroup}







\chapter{UCC::Clubroom}

%\begin{mdframed}
\null
The clubroom is usually open from about 9am until 11pm (staying later than this requires \emph{a good reason} and giving notice to UWA security). It is also sometimes open on weekends, and most days during university holidays. There is a map to the clubroom in Chapter \ref{FindClubroom}. If you want to check the clubroom is open and who is there, check out \url{http://webcam.ucc.asn.au}

The clubroom has a heap of desktop computers, a server room, a projects area with tool cupboard, and assorted storage areas for parts, books and components. These things are all there for members to use --- the locks on everything are to keep thieves out, so please don't be put off, and just ask if you need to get into something.

We don't have cleaners, so look after the clubroom! There is always a door group member in the room, \rant{and you will do what they damn well tell you to do as they are the ones who have to come to cleanups in their free time.} Of course, if you see something that needs doing such as cleaning up, feel free to do it!

\begin{figure}[H]
	\centering
	\includegraphics[width=0.5\textwidth]{figures/webcam.jpg}
	\caption{\url{http://webcam.ucc.asn.au}}
	\label{webcam.jpg}
\end{figure}


%\end{mdframed}
\chapter{Communication Technologies}

% Add Social Media?

\section{Mailing Lists}

UCC has a series of mailing lists that its members use to communicate. 
These include: 

UCC-Announce - An announcement list through which we let 
members know about Events and the like. You are automatically 
subscribed to this list when becoming a member. 

UCC - Having a party that you want to invite UCCans to? This is the 
list for you! Most UCCans subscribe to this list so it's a great 
source of information and discussion on both the club and general 
technical things. 

Committee - The list where committee matters are dealt with and circular 
discussions occur. If you are interested in the running of the club then 
you should subscribe to this list. 

Tech - Technical question? Suggestions or discussion around UCCs setup? 
The list used to discuss UCC's hardware and machines. 

If you want to subscribe to any of these lists, you can do so at 
\url{http://lists.ucc.gu.uwa.edu.au}

\section{IRC}

Without a doubt, the easiest way to waste time in or out of UCC 
is chatting on our Internet Relay Chat (IRC) server. 

You'll get to chat with some of the older members of the club who 
may not even be in Perth. Some of these old guard may seem a 
little grumpy or intimidating at first, but give them a chance, they 
are gold mines for information about the club and all things tech! 
We also have members from CASSA and ComSSA, clubs at other WA unis. 

You can connect with an IRC client to \texttt{irc://irc.ucc.asn.au:6667} 
and join the channel \texttt{\#ucc}, or with a web browser go to 
\url{http://irc.ucc.asn.au}

\zchapter{UCC::Games}\label{Games}

Do you like Steam? Of course you do! Join UCC's Steam Group: \url{http://steamcommunity.com/groups/UCC}

\noindent UCC runs its own TF2, WolfET and Minecraft servers.
There are also a range of other games that members enjoy playing in the clubroom. Here is a summary.

\section{Minecraft}

Minecraft is a game where you mine blocks and craft things out of them. It is more exciting than it sounds. Within 5 minutes you will be addicted. But you probably already know this. 
Our (modded) Minecraft server is at \shell{minecraft.ucc.asn.au}. The modpack changes every three to six months, so you will never get bored!

For more instructions, visit \url{http://minecraft.ucc.asn.au/ucc}. You can also find maps of the world there. %\figref{minecraft_pigmap.png}.
\begin{comment}
\begin{figure}[H]
	\centering
	\includegraphics[width=0.5\textwidth]{figures/minecraft_pigmap.png}
	\caption{Part of the UCC Minecraft world}
	\label{minecraft_pigmap.png}
\end{figure}
\end{comment}

\section{WolfET}

Wolfenstein: Enemy Territory, (or just ET) is an old game but is still incredibly popular at LANs. There is a scholarship \seeref{camp_scholarship} available for the camp which requires playing ET. The ET server is at \server{heathred.ucc.asn.au}.

\section{TF2}

Team Fortress 2, also known as Hat Fortress 2. The TF2 server is located at \shell{heathred.ucc.asn.au}.

\begin{comment}
\begin{figure}[H]
	\centering
	\includegraphics[width=0.5\textwidth]{figures/tf2_hat.png}
	\caption{The ultimate goal of TF2 is to aquire these}
	\label{tf2_hat.png}
\end{figure}
\end{comment}


\section{DoTa}

DoTa is a game that involves a lot of clicking and people yelling about 'Q's and 'W's and 'alties'. Apparently it's fun. UCC doesn't run its own server but it is the most popular game in the clubroom. Unless that's LoL. I can't tell the difference.

\section{LoL}

LoL is a game similar to DoTa but not the same because it is different.

\zchapter{Services}

\newcommand{\server}[1]{\emph{#1}}

\begin{mdframed}
This Chapter provides an overview of UCC's services (as of January 2014); how to use them, what they are for, what servers are responsible for them. The full hostname for a server is \server{server.ucc.asn.au}.

 Servers are usually named after fish beginning with M. This is because they are in the Machine Room, and they run Linux. The mascot for Linux is Tux, a penguin, and he likes to eat fish. The more you know...


Remember that all services are maintained by UCC's members. If you are interested in learning more, or running a new service, ask someone!


\end{mdframed}

\newenvironment{uccservice}[2]
{
	\begin{mdframed}
	\section{#1}
	\begin{mdframed}
		Machine(s) Involved: \server{#2}
	\end{mdframed}

	
}{\end{mdframed}}

\begin{uccservice}{Games}{heathred}

The Heathred A. Loveday memorial games server hosts the following games on a regular basis:
\begin{itemize}
	\item Minecraft -- The server is \server{minecraft}, a VM on \server{heathred}
	\item TF2
	\item Enemy Territory (popular at LANs)
\end{itemize}


Administrator access to \server{heathred} is fairly unrestricted; it is also available as a general use server. For example, its GPU has been used in the past for number crunching projects.

\end{uccservice}



\begin{uccservice}{Drinks and Snacks --- Dispense}{merlo, coke machine, snack machine}

UCC's most successful service is undoubtably the internet connected coke machine and not quite internet connected snack machine. These use serial communications to talk to \server{merlo}, which runs open source software written by talented members including John Hodge, Mark Tearle and David Adam. 

A relay connected to \server{merlo} can be activated by door members from the snack machine to open the club's electronic door lock.

\end{uccservice}

\pagebreak

\begin{uccservice}{Mumble}{heathred}

What's that? I couldn't quite hear you?

Mumble is a thing for voice chat whilst playing games. \server{heathred} runs a surprisingly popular Mumble server.

\end{uccservice}

\begin{uccservice}{Clubroom Music}{robotnik}

From within the clubroom, you can navigate to \url{http://robotnik} to play music over the speakers. Beware, as repeated abuse may lead to activation of the dreaded "loldongs" mode.

\end{uccservice}

\begin{uccservice}{Email}{mooneye}

UCC proudly runs its own mail server. You have an email account <username@ucc.asn.au>. The address <username@ucc.gu.uwa.edu.au> will also work.

Upon creating your account you can choose an address to foward all emails to. You can change this at any time by editing the ".forward" file in your home directory.

Alternately, you can use one of several methods to check your UCC email directly.
\begin{enumerate}
	\item alpine --- Connect via SSH and run "alpine". The interface is basic. Press the Control key a lot.
	\item webmail --- Several options will be presented to you at \url{http://webmail.ucc.asn.au}
	\item mail client (eg: Thunderbird) --- The server name is \server{secure.ucc.asn.au}. Use port 993 and IMAP. With your UCC username and password.
\end{enumerate}

\end{uccservice}

\begin{uccservice}{Web Hosting}{mantis, mussel}

Members can publish their own sites! SSH to a server and edit the files in the directory "public-html". The website will appear at \url{http://username.ucc.asn.au}.

\end{uccservice}

\begin{uccservice}{Wiki Hosting}{mooneye}

UCC uses a Wiki called "MoinMoin" to store documentation on servers, events, and miscellaneous things. It is visible at \url{http://wiki.ucc.asn.au}.

\end{uccservice}

\begin{uccservice}{User Logins}{mussel, mylah}

We use something called LDAP for authentication and linux accounts. SAMBA is involved for windows logins. Only one member really knows how this works, so I will move swiftly on.

\end{uccservice}

\begin{uccservice}{File Storage}{mylah, enron/stearns, nortel/onetel, motsugo}

With your account comes not one, but \emph{two} directories for your files.

The one most commonly seen is accessable on clubroom machines. It will be named "/home/ucc/username" on clubroom linux machines. On servers however, there is a different "/home" directory; to get to your clubroom home directory you must access "/away/ucc/username".

Home directories on the servers are considered slightly more secure than your "away" directory.

If you are using Linux, you can use the program "sshfs" to mount your home or away directories remotely. This is the most convenient method for uploading, downloading or editing files. Under windows, the programs "WinSCP" or "Filezilla" might not suck.

%For interest: Not that interesting
%\server{enron} and \server{stearns} are our slowly dying SAN which stores "away". \server{mylah} mounts the SAN directly and exports the filesystem over NFS.
%\server{motsugo}'s disks contain "home" which is exported only to servers via NFS.
%The NetApp \server{nortel} and \server{onetel} store Virtual Machine (VM) images, and "/services" --- the directory that contains UCC's website, amongst other things.

\end{uccservice}

\begin{uccservice}{Network Servers}{murasoi, mooneye}

Murasoi is a wheel-only server which serves as a router for all of UCC's networks and runs the infamous "ucc-fw" firewall. Murasoi also acts as the DHCP server.

For some reason, DNS is not on murasoi, but on mooneye. The magic that makes \url{http://username.ucc.asn.au} point to your website happens on mooneye.

\end{uccservice}

\begin{uccservice}{Virtual Machine Hosting}{medico, motsugo, heathred, mylah}

Members who are particularly nice to wheel group can get their own VM hosted at UCC. \server{medico} runs the amazing ProxMox interface and is used for all new VMs. The typical way to use this interface is from a web browser on \server{maaxen}, a VM running on \server{medico}...

\server{heathred} is used for VMs when wheel complains that they aren't important enough to justify using all of \server{medico}'s CPU *cough* minecraft *cough*.

\end{uccservice}

\begin{uccservice}{Windows Server}{maaxen}

\server{maaxen} is our token Windows server. It can be accessed through RDP, but beware, as it only supports two simultaneous sessions. \server{maaxen} boasts a range of useful programs including Notepad and Matlab.

\end{uccservice}

\begin{uccservice}{IRC}{mussel, mantis}

IRC is discussed in Chapters \ref{SetupAccount} and \ref{Communications}.

Our two IRC servers are bridged with CASSA and ComSSA, computer science associations at other Universities.

\end{uccservice}

\begin{uccservice}{General Use}{motsugo}

SSH access is available to several servers, but \server{motsugo} is the best choice for general use. It is mostly used for personal software projects, and to run members' screen sessions so they can be \emph{constantly} connected to IRC.

\end{uccservice}



\linespread{1.0}
\zchapter{Common UNIX/Linux Commands}\label{UnixCommands}

% Whenever a chapter is started with a figure or table, it forces a blank page?
% -> Add some text
In the Beginning was the Command Line.

%\begin{mdframed}

%A large number of the UCC's computers run UNIX or Linux. If you've never encountered UNIX before, it might be a bit daunting for you. While most UNIX operating systems come with nice graphical desktops, the power is all in the text-based shell. Here are some common shell commands, in no particular order. There will be a "Learn to Linux" night at some point to help you out with some of the finer points.

%Many commands have an available help summary that you can get by appending \verb/--help/.

%\end{mdframed}

\newcommand{\uc}[2]{ \shell{#1} & \small{#2} \\}

\centering
	\rowcolors{1}{}{lightgray}
	\begin{tabular}{|p{0.35\linewidth}|p{0.6\linewidth}|}
		\rowcolor{black} \color{white}{Command} & \color{white}{Description} \\



		\uc{logout}{Logs you off the system. Do this before you leave}
		\uc{ls <directory>}{Lists the files in the given directory (folder)}
		\uc{cd <directory>}{Change to the given directory (folder)}
		\uc{mkdir <directory>}{Add the specified directory (folder)}
		\uc{rmdir <directory>}{Remove the specified directory (folder)}
		\uc{pwd}{"Print Working Directory" --- Displays the path of the directory you are currently in}
		\uc{less <file>}{Read through a file (\emph{space} scrolls, \emph{q} quits)}
		\uc{cp <source> <destination>}{Make a copy of a file in a new place}
		\uc{mv <source> <destination>}{Move (rename) a file}
		\uc{rm <file>}{\textsc{{Permanently deletes a file. \newline There is no "Recycle Bin"!}}}
%		\uc{alpine}{This program can be used for reading emails}
		\uc{nano <file>, vim <file>, emacs <file>}{Three different text file editors. \shell{nano} is the simplest}
		\uc{finger <username>}{Check to see who someone is and if they are logged in}
		\uc{tla <username|TLA>}{Look up a UCC TLA}
		\uc{ssh <username>@<hostname>}{Log in (securely) to another (UNIX) machine}
		\uc{ping <hostname>}{Ping another machine to see if it is up and what the latency is. Press \emph{Ctrl-C} to cancel}
		\uc{man <command>}{Displays the manual for a command. See \shell{man man} for more information}
		\uc{top}{Displays an updating list of current processes} 
%		\uc{ps}{Lists the processes you are running on this terminal (names and PIDs)}
		\uc{ps aux}{Lists all processes running on the system}
		\uc{kill <PID>}{Tell a process it should stop. \shell{kill -9} will kill a process immediately}
		\uc{passwd}{On \server{mussel} - Change your login password}
		\uc{dispense}{Get yourself a delicious drink!}
		\uc{gcc/g++}{C/C++ compilers for compiling some awesome code you wrote}
		\uc{python}{Python interpreter for interpreting some OK code you wrote}
		\uc{irssi}{Join IRC to chat/argue with other members}
		\hline
	\end{tabular}





\chapter*{Glossary}

Words and phrases that are (or were) useful around the club.

\begin{tabular}{ll}
	{\bf Term} & {\bf Meaning} \\
Dr ACC Murphy &
A Computer Called Murphy. Dr ACC Murphy is infamous around the UCC. He even receives mail! \\
BSD &
Berkeley Systems Distribution – a UNIX developed at Berkeley, now better known through the FreeBSD, NetBSD and OpenBSD UNIXes. musdea runs FreeBSD. \\
blog &
aka. weblog – sort of like a journal on the Internet (you don't have one?). Think of it as  a tumblr for grown-ups. Syndicated by a Planet. \\
Coke credit & 
If you gotta ask, you ain't got it! Coke credit is how people usually refer to money in your dispense account. \\
Coke Group &
The people who can put money (Coke credit) in your dispense account. \\
Debian &
a Linux distribution popular in the UCC due to its community nature. \\
DEC Terminal &
A dumb serial terminal, useful for plugging into the serial console on servers (possibly via a terminal server). Has a model number like vt100, vt200 or vt420. There's one in the corridor you can log into mussel and dispense drinks from. \\
dispense &
dispense started off as a way to dispense Cokes from the online Coke machine, and has since grown into the way UCCans think the world should do business. \\
Door Group &
the group of people charged with keep the room open, tidy and safe. \\
Firefox &
A web browser by the Mozilla Foundation, arguably the second-worst Internet browser – the worst is every other browser. \\
Flame &
Flame is the UCC's MUD (Multi-User Dungeon) and may or may not be haunted. \\
Fresher &
A new university student, usually also a first time UCC member. \\
Fresher Rep &
Fresher Committee member, usually chosen because they look like they'll make a good worker drone in the future. Represents the freshers at committee meetings, if they attend. \\
GNOME &
GNU Networked Object Model Environment – an open source desktop environment aimed primarily at UNIX computers. Quite popular in the past, before it removed all of the features people liked. \\
GNU &
GNU is Not Unix – a layer of libraries and utilities to implement a UNIX like operating system, commonly used on top of Linux. \\
Internet Explorer &
Just use Firefox. No, really. \\
IRC &
The lifeblood of UCCan communication. Internet Relay Chat lets you share news, stories and terrible puns with other members worldwide. \\
KDE &
The K Desktop Environment. A rival of GNOME which some members prefer. Contains a frankly scary number of features, and a memory footprint to match. \\
kernel &
The core of an operating system. All operating systems have a kernel, some popular ones include the Linux kernel and the Mach kernel. \\
LDAP &
Lightweight Directory Access Protocol: the black magic used for authentication at the UCC. \\
Linux &
the kernel (basis) of an open source UNIX operating system that has developed quite a following among computer scientists and engineers. \\
loft &
the area above the UCC that looks down into the UCC clubroom. LAN gaming and other activities take place up there. \\
machine room &
The UCC data centre. This is the small room with the glass doors that is located within the clubroom. All of our servers are kept in this room. It is locked when there is no one from Wheel around. \\
mailing list &
a way of communicating with a very large number of people via email. The UCC has several mailing lists of varying popularity. \\
Mozilla &
develop several open source web related products, such as Firefox and Thunderbird. \\
OCM &
Ordinary Committee Member – the worker drones of the UCC Committee, they do lots of work, for little reward. \\
Oligoboot &
boots more then one operating system (selectable when you boot). \\
OpenSolaris &
A UNIX developed by Sun Microsystems, and abandoned by Oracle. \\
NeXTStep &
An operating system developed by NeXT before they were bought out by Apple. Lots of NeXTStep is incorporated into Mac OS X. \\
open source &
A software ideology, where the source code to software (what is compiled into the program you run) is freely available. Also known as Free Software, exactly what makes a program open source is a good way to get into an argument. \\
Planet &
A web page that syndicates blogs. UCC has one at \url{http://planet.ucc.asn.au/} \\
Secret Wheel Song &
The song that is supposedly sung at the beginning of each Wheel meeting. \\ % Elitism alert...
SLA &
A Service Level Agreement. A document describing how reliable services are. UCC does not have one, and you should not ask Wheel members about it. \\
terminal server &
Sort of a router for serial ports, allows you to connect to one serial port from another. Usually connected to DEC Terminals, servers and dispense. (It can also refer to other sorts of servers which provide login sessions over the network). \\
theft book &
this is where you write down that you borrowed tools from UCC. It is not for borrowing books; you mail books@ucc.asn.au to do that. \\
TLA &
Three Letter Acronym – a way to refer to UCC members, often used in the minutes of meetings. Use the tla program or visit the UCC website to decode them. \\
Ubuntu &
A Linux distribution derived from Debian. Funded with space money. \\
UCCan &
someone who spends a lot of time in the UCC. Some UCCans pass their units. \\
Unifi &
The University wireless network. Available around campus where UCC's wireless is not. You need to be a student to access it though. \\
UniSFA &
the University Science Fiction Association, the ones down the hall. \\
VM &
A VM or Virtual Machine is a computer emulated on top of another computer. Many of the UCC's servers run inside VMs. The claim is that UCC's VMs are not a ``single point of  failure.`` That's the machine they are all running on (medico). \\
WAIX &
WA Internet eXchange – a group of ISPs and interested bodies who peer resources on the Internet for mutual benefit. \\
Wheel Group &
the group responsible for maintaining computers, accounts and services in UCC. \\
\end{tabular}

% ETC

%\newpage
%---------------------------------------------------------
%\renewcommand{\bibname}{References}
%\bibliography{references/refs}
%\bibliographystyle{ieeetr}  
%\addcontentsline{toc}{part}{References}
%---------------------------------------------------------

% Appendices
%\appendix
%\chapter*{Glossary}

Words and phrases that are (or were) useful around the club.

\begin{tabular}{ll}
	{\bf Term} & {\bf Meaning} \\
Dr ACC Murphy &
A Computer Called Murphy. Dr ACC Murphy is infamous around the UCC. He even receives mail! \\
BSD &
Berkeley Systems Distribution – a UNIX developed at Berkeley, now better known through the FreeBSD, NetBSD and OpenBSD UNIXes. musdea runs FreeBSD. \\
blog &
aka. weblog – sort of like a journal on the Internet (you don't have one?). Think of it as  a tumblr for grown-ups. Syndicated by a Planet. \\
Coke credit & 
If you gotta ask, you ain't got it! Coke credit is how people usually refer to money in your dispense account. \\
Coke Group &
The people who can put money (Coke credit) in your dispense account. \\
Debian &
a Linux distribution popular in the UCC due to its community nature. \\
DEC Terminal &
A dumb serial terminal, useful for plugging into the serial console on servers (possibly via a terminal server). Has a model number like vt100, vt200 or vt420. There's one in the corridor you can log into mussel and dispense drinks from. \\
dispense &
dispense started off as a way to dispense Cokes from the online Coke machine, and has since grown into the way UCCans think the world should do business. \\
Door Group &
the group of people charged with keep the room open, tidy and safe. \\
Firefox &
A web browser by the Mozilla Foundation, arguably the second-worst Internet browser – the worst is every other browser. \\
Flame &
Flame is the UCC's MUD (Multi-User Dungeon) and may or may not be haunted. \\
Fresher &
A new university student, usually also a first time UCC member. \\
Fresher Rep &
Fresher Committee member, usually chosen because they look like they'll make a good worker drone in the future. Represents the freshers at committee meetings, if they attend. \\
GNOME &
GNU Networked Object Model Environment – an open source desktop environment aimed primarily at UNIX computers. Quite popular in the past, before it removed all of the features people liked. \\
GNU &
GNU is Not Unix – a layer of libraries and utilities to implement a UNIX like operating system, commonly used on top of Linux. \\
Internet Explorer &
Just use Firefox. No, really. \\
IRC &
The lifeblood of UCCan communication. Internet Relay Chat lets you share news, stories and terrible puns with other members worldwide. \\
KDE &
The K Desktop Environment. A rival of GNOME which some members prefer. Contains a frankly scary number of features, and a memory footprint to match. \\
kernel &
The core of an operating system. All operating systems have a kernel, some popular ones include the Linux kernel and the Mach kernel. \\
LDAP &
Lightweight Directory Access Protocol: the black magic used for authentication at the UCC. \\
Linux &
the kernel (basis) of an open source UNIX operating system that has developed quite a following among computer scientists and engineers. \\
loft &
the area above the UCC that looks down into the UCC clubroom. LAN gaming and other activities take place up there. \\
machine room &
The UCC data centre. This is the small room with the glass doors that is located within the clubroom. All of our servers are kept in this room. It is locked when there is no one from Wheel around. \\
mailing list &
a way of communicating with a very large number of people via email. The UCC has several mailing lists of varying popularity. \\
Mozilla &
develop several open source web related products, such as Firefox and Thunderbird. \\
OCM &
Ordinary Committee Member – the worker drones of the UCC Committee, they do lots of work, for little reward. \\
Oligoboot &
boots more then one operating system (selectable when you boot). \\
OpenSolaris &
A UNIX developed by Sun Microsystems, and abandoned by Oracle. \\
NeXTStep &
An operating system developed by NeXT before they were bought out by Apple. Lots of NeXTStep is incorporated into Mac OS X. \\
open source &
A software ideology, where the source code to software (what is compiled into the program you run) is freely available. Also known as Free Software, exactly what makes a program open source is a good way to get into an argument. \\
Planet &
A web page that syndicates blogs. UCC has one at \url{http://planet.ucc.asn.au/} \\
Secret Wheel Song &
The song that is supposedly sung at the beginning of each Wheel meeting. \\ % Elitism alert...
SLA &
A Service Level Agreement. A document describing how reliable services are. UCC does not have one, and you should not ask Wheel members about it. \\
terminal server &
Sort of a router for serial ports, allows you to connect to one serial port from another. Usually connected to DEC Terminals, servers and dispense. (It can also refer to other sorts of servers which provide login sessions over the network). \\
theft book &
this is where you write down that you borrowed tools from UCC. It is not for borrowing books; you mail books@ucc.asn.au to do that. \\
TLA &
Three Letter Acronym – a way to refer to UCC members, often used in the minutes of meetings. Use the tla program or visit the UCC website to decode them. \\
Ubuntu &
A Linux distribution derived from Debian. Funded with space money. \\
UCCan &
someone who spends a lot of time in the UCC. Some UCCans pass their units. \\
Unifi &
The University wireless network. Available around campus where UCC's wireless is not. You need to be a student to access it though. \\
UniSFA &
the University Science Fiction Association, the ones down the hall. \\
VM &
A VM or Virtual Machine is a computer emulated on top of another computer. Many of the UCC's servers run inside VMs. The claim is that UCC's VMs are not a ``single point of  failure.`` That's the machine they are all running on (medico). \\
WAIX &
WA Internet eXchange – a group of ISPs and interested bodies who peer resources on the Internet for mutual benefit. \\
Wheel Group &
the group responsible for maintaining computers, accounts and services in UCC. \\
\end{tabular}

%\include{appendices/figures}

%---------------------------------------------------------

\end{document}

